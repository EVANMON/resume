% !TEX TS-program = xelatex
% !TEX encoding = UTF-8 Unicode
% !Mode:: "TeX:UTF-8"

\documentclass{resume}
\usepackage{zh_CN-Adobefonts_external} % Simplified Chinese Support using external fonts (./fonts/zh_CN-Adobe/)
%\usepackage{zh_CN-Adobefonts_internal} % Simplified Chinese Support using system fonts
\usepackage{linespacing_fix} % disable extra space before next section
\usepackage{geometry}
\usepackage{cite}
\geometry{a4paper, top=10mm}

\begin{document}
\pagenumbering{gobble} % suppress displaying page number
\name{\heiti{张仲霖}}

\basicInfo{
  \email{zhonglin\_zhang@outlook.com} \textperiodcentered\ 
  \phone{(+1) 424-333-8171} \textperiodcentered\ 
  \github{https://github.com/EVANMON}}
\basicInfo{\faHome{10933 Rochester Ave, Apt 109, Los Angeles, CA 90024, USA}}
 % \linkedin[billryan8]{https://www.linkedin.com/in/billryan8}}
 
\section{\faGraduationCap\  教育背景}
\datedsubsection{\textbf{加州大学洛杉矶分校}, 洛杉矶}{2017年9月 -- 2019年3月(预计)}
\textit{在读硕士}\ 电子与计算机工程,GPA: 3.85 / 4.0 \\ 相关课程: 机器学习导论, 大规模数据挖掘, 无线与移动网络协议及系统设计
\datedsubsection{\textbf{上海交通大学}, 上海}{2013年9月 -- 2017年6月}
\textit{学士}\ 信息工程,GPA: 3.60 / 4.0 \\ 相关课程: 数据结构与算法, C++编程, Python编程基础, 操作系统, 数字语音处理, 嵌入式系统

\section{\faUsers\ 项目经历}
\datedsubsection{\textbf{阿里妈妈国际广告算法大赛}(\textit{Python, Machine Learning})}{2018年3月 -- 至今}
%\role{实习}{经理: 高富帅}
%xxx后端开发
\begin{itemize}
  \item 利用淘宝平台提供的近48万交易数据对4种学习模型进行训练,预测搜索广告的转换概率
  \item 训练中提取的特征包括广告商品,用户信息,网页上下文以及网店信息
  \item 计算预测结果的Logarithmic Loss作为训练评估指标,目前已取得的最低值为0.08240
\end{itemize}

\datedsubsection{\textbf{数据挖掘模型及算法研究}(\textit{Python, Machine Learning})}{2018年1月 -- 2018年3月}
%\role{Golang, Linux}{个人项目,和富帅糕合作开发}
%\begin{onehalfspacing}
%分布式负载均衡科学上网姿势, https://github.com/cyfdecyf/cow
\begin{itemize}
  \item 对\textit{20 Newsgroups}数据集中的18846篇文章建立TF-IDF矩阵并使用k-Means聚类算法复原标签
  \item 利用协同过滤方法处理关于9125部电影的100004条用户评分,实现电影推荐系统并检验准确性
  \item 通过学习一个网络备份系统的18588条记录,使用回归算法(Neural Network, k-NearestNeighbor, Polynomial Function Model)预测备份资料的规模
\end{itemize}
%\end{onehalfspacing}

\datedsubsection{\textbf{手机游戏的网络延迟研究}(\textit{Python, LTE})}{2017年10月 -- 2017年12月}
%\role{\LaTeX, Python}{个人项目}
%\begin{onehalfspacing}
%优雅的 \LaTeX\ 简历模板, https://github.com/billryan/resume
\begin{itemize}
  \item 利用Wireshark收集安卓平台的6个手机游戏的传输数据并研究其通讯规律
  \item 负责上行传输过程的延迟分析,具体工作包括:
  \begin{itemize}
    \item 使用MobileInsight记录LTE模式下手机游戏的数据包,实现Python程序对超过30000条数据进行批量处理,量化过程中出现的时延,并单独分析其对总时延的影响
    \item 在研究中,网络质量提高15dBm时可以降低大约78.8\%的上行传输时延
  \end{itemize}
  %\item 支持 FontAwesome 4.5.0
\end{itemize}
%\end{onehalfspacing}

\datedsubsection{\textbf{全息增强现实(AR)智能眼镜设计}(\textit{AR, Java})}{2015年10月 -- 2016年10月}
%\role{Golang, Linux}{个人项目,和富帅糕合作开发}
%\begin{onehalfspacing}
%分布式负载均衡科学上网姿势, https://github.com/cyfdecyf/cow
\begin{itemize}
  \item 在安卓平台实现基于肤色的手势识别和跟踪算法,并开发其相关AR应用
  \item 设计基于全息图原理的AR智能显示光学结构,使用COMSOL仿真软件进行仿真并优化设计参数
\end{itemize}
% Reference Test
%\datedsubsection{\textbf{Paper Title\cite{zaharia2012resilient}}}{May. 2015}
%An xxx optimized for xxx\cite{verma2015large}
%\begin{itemize}
%  \item main contribution
%\end{itemize}

\section{\faCogs\ 技能与兴趣}
% increase linespacing [parsep=0.5ex]
\begin{itemize}[parsep=0.5ex]
  \item 编程语言: C++ == Python > Java
  \item API / 开发工具: Scikit-learn, Surprise, IPython, Pycharm, Wireshark, Github 
  \item 研究兴趣: 机器学习,数据挖掘,AR/VR显示,计算机网络系统 
\end{itemize}

\section{\faHeartO\ 获奖情况}
\datedline{专业优秀奖学金 (25/180)}{2015年10月}
\datedline{校级奖学金 (20/180)}{2014年10月}

\section{\faInfo\ 兴趣与公益活动}
% increase linespacing [parsep=0.5ex]
\begin{itemize}[parsep=0.5ex]
  \item 个人兴趣: 排球,合唱,电影,侦探小说
  \item \datedline{Gapper国际义工项目 \textit{斯里兰卡}}{2015年7月}
  \begin{itemize}[parsep=0.5ex]
    \item 在汉班托塔一所当地中学实用英语为两个年级的学生教授自然科学
  \end{itemize}  
\end{itemize}

%% Reference
%\newpage
%\bibliographystyle{IEEETran}
%\bibliography{mycite}
\end{document}
